% +--------------------------------------------------+
% | Typeset this file to document enumitem.sty       |
% +--------------------------------------------------+
%
% Copyright (c) 2003-2011 by Javier Bezos.
% All Rights Reserved.
%
% This file is part of the enumitem distribution release 3.5.2
% ------------------------------------------------------------
% 
% It may be distributed and/or modified under the
% conditions of the LaTeX Project Public License, either version 1.3
% of this license or (at your option) any later version.
% The latest version of this license is in
%   http://www.latex-project.org/lppl.txt
% and version 1.3 or later is part of all distributions of LaTeX
% version 2003/12/01 or later.
% 
% This work has the LPPL maintenance status "maintained".
% 
% The Current Maintainer of this work is Javier Bezos.

\documentclass[a4paper]{ltxguide}

\makeatletter
\newenvironment{desc}
  {\if@nobreak
     \vskip-\lastskip
     \vspace*{-2.5ex}%
   \fi
   \decl}
  {\enddecl}
\makeatother

\newcommand\3{\unskip\enspace\fbox{\fontsize{4}{4}\selectfont NEW 3.0}}

\usepackage{hyperref}

%\usepackage{pslatex}

\title{Customizing lists\\with the\\\textsf{enumitem} package}

\author{Javier Bezos\footnote{For bug reports, comments and
suggestions go to \href{http://www.tex-tipografia.com/enumitem.html}%
{\texttt{http://www.tex-tipografia.com/enumitem.html}}.
English is not my strong point, so contact me when you
find mistakes in the manual. Other packages by the same author:
\textsf{gloss} (with Jos\'e Luis D\'{\i}az), \textsf{accents,
tensind, esindex, dotlessi, titlesec, titletoc}.}}

\date{Version 3.5.2\\2011-09-28}

\IfFileExists{enumitem.sty}{\usepackage{enumitem}}{}

\addtolength{\topmargin}{-3pc}
\addtolength{\textwidth}{6pc}
\addtolength{\oddsidemargin}{-2pc}
\addtolength{\textheight}{7pc}

\raggedright
\parindent1.8em
\parskip0pt

\begin{document}

\maketitle

\section{Introduction}

When I began to use \LaTeX{} several year ago, two particular points
annoyed me because I found customizing them was very complicated
---headlines/footlines and lists.  A new way to redefine the former is
accomplished in my own \textsf{titlesec} package, but none was
available to customize the latter except:
\begin{itemize}
\item \textsf{enumerate}, which just allows to change the label and
it does it pretty well.

\item \textsf{mdwlist}, which only ``provides some vaguely useful
list-related commands and environments,'' as its manual states,
and not a coherent way of handling lists.

\item \textsf{paralist}, which provides lists within a paragraph (the
original purpose of this package), a few other hard-wired
specific changes and the optional argument of \textsf{enumerate}.
\end{itemize}

One of the main drawbacks of the standard |list| is its weird
parameters, whose meaning is not always obvious.  In order to provide
a cleaner interface two approaches were possible: either defining new
lists, or introducing a new syntax making the standard lists easier to
customize.  For marks I took the first approach in titlesec, just
because I did not manage to find a satisfactory solution with the
\LaTeX{} internal macros, but since lists are in some sense more
``complete'' than sections and marks, I have taken here the second
approach.

In the interface a sort of ``inheritance'' is used. You can
set globally the behaviour of lists and then override several
parameters of, say, enumerate and then in turn override
a few paremeters of a particular instance. The values will
be searched in the hierarchy.

\section{The package}

This package is intended to ease customizing the three
basic list environments: |enumerate|, |itemize| and
|description|. It extends their syntax to allow
an optional argument where a set of parameters in the
form |key=value| are available:
\begin{itemize}
\item
Vertical spacing:
\begin{itemize}
\item |topsep|
\item |partopsep|
\item |parsep|
\item |itemsep|
\end{itemize}
\item
Horizontal spacing:
\begin{itemize}
\item |leftmargin|
\item |rightmargin|
\item |listparindent|
\item |labelwidth|
\item |labelsep|
\item |itemindent|
\end{itemize}
\end{itemize}

For example:
\begin{verbatim}
\begin{itemize}[itemsep=1ex,leftmargin=1cm]
\end{verbatim}

The keys above are equivalent to the well known list parameters---see a
\LaTeX{} manual for a description of them.  Next sections explains the
extensions provided by |enumitem|.

\section{Keys}

This section describes the keys in displayed lists. Most of them are 
available in inline lists, where further keys are available (see 
\ref{s.inline}).

\subsection{Label and cross references format}

\begin{desc}
|label=<commands>|
\end{desc}

Sets the label to be used in the current level.
A set of starred versions of |\alph|, |\Alph|,
|\arabic|, |\roman| and |\Roman|, without argument
stand for the current counter in |enumerate|.\footnote{Actually,
the asterisk is currently the argument but things may change.
Consider them as starred variants and follow the corresponding
syntax.} Thus
\begin{verbatim}
\begin{enumerate}[label=\emph{\alph*})]
\end{verbatim}
prints \textit{a}), \textit{b}), and so on (this is a
standard style in Spanish, and formerly used by Chicago, too).

It works with |\value|, too (provided the widest label is not to be
computed or \verb|widest*| is used, see below).  A fancier example
(which looks ugly, but it is intended only to illustrate what is
possible; requires \textsf{color} and \textsf{pifont}):
\begin{verbatim}
\begin{enumerate}[label=\protect\fcolorbox{blue}{yellow}{\protect\ding{\value*}},
                  start=172]
\end{verbatim}

The value of |label| is a moving argument, and fragile commands must
be protected \textit{except} the counters.  Because of that, use of
\verb|\value| is somewhat tricky, because \verb|\the| or \verb|\ifnum|
expects an actual value, which is not the case when \verb|label| is
being processed to replace internally the \verb|*| by the form with
the counter argument.  The best solution is usually encapsulating the
logic inside a new ``counter'' with the help of
\verb|\AddEnumerateCounter|.\footnote{Which is admittedly somewhat
convoluted.  A better way to accomplish this is on the way.}

If you prefer setting labels like the \textsf{enumerate} package, use 
``short labels'' (see section \ref{s.short}).

\begin{desc}
|label*=<commands>|
\end{desc}

Like |label| but its value is appended to the parent label.  For example, the follollowing defines a |legal| list (1.,
1.1., 1.1.1., and so on):
\begin{verbatim}
\newlist{legal}{enumerate}{10}
\setlist[legal]{label*=\arabic*.}
\end{verbatim}

\begin{desc}
|ref=<commands>|
\end{desc}

By default, |label| sets also the form of cross references and
|\the...| (overriding the settings in previous hierarchical
levels), but you can define a different format with this key.  For
example, to remove the right parethesis:
\begin{verbatim}
\begin{enumerate}[label=\emph{\alph*}),ref=\emph{\alph*}]
\end{verbatim}
In both |label| and |ref|, the counters can be
used as usual:
\begin{verbatim}
\begin{enumerate}[label=\theenumi.\arabic*.]
\end{verbatim}
or
\begin{verbatim}
\begin{enumerate}[label=\arabic{enumi}.\arabic*.]
\end{verbatim}
(provided the current level is the second one).

Note the |label|s are \textit{not} accumulated to form
the reference. If you want, say, something like 1.\textit{a}
from 1) as first level and \textit{a}) as second level,
you must set it with |ref|. You may use
|\ref{level1}.\ref{level2}| with appropiate |ref|
settings, but as Robin Fairbairns points out in the \TeX{} FAQ
\begin{quote}
\dots{} [that] would be both tedious and error-prone. What is more, it 
would be undesirable, since you would be constructing a visual 
representation which is inflexible (you could not change all the 
references to elements of a list at one fell swoop).
\end{quote}
This is sensible and I recommend to follow the advice, but sometimes
you might want something like:
\begin{verbatim}
... subitem \ref{level2} of item \ref{level1} ...
\end{verbatim}

The value of |ref| is a moving argument, and fragile
commands must be protected \textit{except} the counters.

\begin{desc}
|font=<commands>|\qquad\verb|format=<commands>|
\end{desc}

Sets the label font.  Useful when the label is changed with the
optional argument of |\item| and in \texttt{description}.  The last
command in |<commands>| can take an argument with the item label.  In
\texttt{description} class setting are in force, so you may want begin
with \verb|\normalfont|. A synonymous is \texttt{format}.

\begin{desc}
|align=left|\qquad |align=right|\qquad |align=parleft|\3
\end{desc}

How the label is aligned (with relation to the label box edges).
Three values are possible: |left|, the default |right| and
\verb|parleft| (a parbox of width \verb|\labelwidth| with flush left
text).  The parameters controlling the label spacing should be
properly set, either by hand or more conveniently with the |*|
settings (see below):
\begin{verbatim}
\begin{enumerate}[label=\Roman*., align=left, leftmargin=*]
\end{verbatim}
(When the label box is supposed to have its natural width, use
|left|.) 

\begin{desc}
|\SetLabelAlign{<value>}{<commands>}|\3
\end{desc}

New align types can be defined (or the existing ones redefined) with
|\SetLabelAlign|; the predefined values are equivalent
to:\footnote{Prior to version 3.0 the left alignments was incorrectly
defined and the label and the text could overlap.}
\begin{verbatim}
\SetLabelAlign{right}{\hss\llap{#1}}
\SetLabelAlign{left}{#1\hfil}
\SetLabelAlign{parleft}{\strut\smash{\parbox[t]\labelwidth{\raggedright##1}}}
\end{verbatim}

If the last thing in the definition is a skip (typically \verb|\hfil|), it is
removed sometimes by description. If for some reason you want to avoid
this, just add \verb|\null| at the end.

Although primarily intended for the alignment, this commands has other
uses (as in the provided \verb|parleft|).  For example, with the
following all labels with |align=right| are set as superscripts:
\begin{verbatim}
\SetLabelAlign{right}{\hss\llap{\textsuperscript{#1}}}
\end{verbatim}
(A new name is also possible, of course.)

If you want the internal settings for \texttt{align} and \texttt{font}
be ignored, you can override the \textsf{enumitem} definition of
\verb|\makelabel| in \texttt{before}:
\begin{verbatim}
\begin{description}[before={\renewcommand\makelabel[1]{\ref{##1}}}]
\end{verbatim}
(Alternatively, define a macro and use \verb|\let|.) 

\subsection{Horizontal spacing of labels}

\begin{desc}
|labelindent=<length>|\\
\verb|\labelindent|
\end{desc}

This  parameter is added in \textsf{enumitem} for the blank space from
the margin of the enclosing list/text to the left edge of the label box.  This
means there is a redundancy because one of the parameters depends on
the others, i.e., it has to be computed from the other values, as
described below.  There is a new counter length |\labelindent| which
defaults to 0 pt. The five parameteres are related in the following 
way:
\[
\verb|\leftmargin|+\verb|\itemindent| = 
\verb|\labelindent|+\verb|\labelwidth|+\verb|\labelsep|
\]

\begin{desc}
|leftmargin=!|\qquad|itemindent=!|\qquad|labelsep=!|
\qquad|labelwidth=!|\qquad|labelindent=!|\3
\end{desc}

Sets which value is to be computed from the others. This is done 
after \textit{all} keys has been read. Explicit values are not lost, and
so with the following hierarchical settings:
\begin{verbatim}
leftmargin=2em
labelindent=1em,leftmargin=!
labelindent=!
\end{verbatim}
|leftmargin| is again 2em and |labelindent| is the computed parameter.
The default is |labelindent=!|, but note some keys sets another value 
(\verb|wide| and description \verb|style|s).

With |align=right| (the default), |labelindent=!| and
|labelwidth=!| behave similarly in practice. 

\begin{desc}
|leftmargin=*|\qquad|itemindent=*|\qquad|labelsep=*|
\qquad|labelwidth=*|\qquad|labelindent=*|
\end{desc}

Like before, but |labelwidth| is set to the width of the current
label, using the default value of \textit{0} in |\arabic*|,
\textit{viii} in |\roman*|, \textit{m} in |\alph*| and
similarly in uppercase forms (these values can be changed with
|widest|, see below).  Examples are:
\begin{verbatim}
\begin{itemize}[label=\textbullet, leftmargin=*]
\begin{enumerate}[label=\roman*), leftmargin=*, widest=iii]
\begin{itemize}[label=\textbullet,
                leftmargin=2pc, labelsep=*]
\begin{enumerate}[label=\arabic*., leftmargin=2\parindent, 
                  labelindent=\parindent, labelsep=*]
\end{verbatim}

The most useful are |labelsep=*| and |leftmargin=*|.
With the former the item body begins at a fixed place (namely,
|leftmargin|), while with the latter begins at a variable place
depending on the label (but always the same within a list, of course).
Most of times, what you would want is |leftmargin=*|.

Unfortunately, \LaTeX{} does not define a default |labelsep| to
be applied to all lists---simply the current value is used.  With
\textsf{enumitem} you can set default values for every list, as
described below, and so, if you want to make sure |labelsep| is
under your control, all you need is something like:
\begin{verbatim}
\setlist{itemsep=.5em}
\end{verbatim}

|labelwidth=*| and |labelwidth=!| are synonymous.

\begin{desc}
|widest=<string>|\qquad|widest*=<integer>|\3\qquad|widest|
\end{desc}

To be used in conjunction with the \texttt{*}-values, if
desired.  It overrides the default value for the widest
printed counter.  Sometimes, if lists are not very long, a value of
|a| for |\alph| is more sensible than the default |m|:
\begin{verbatim}
\begin{enumerate}[leftmargin=*,widest=a] % Assume standard 2nd level
\end{verbatim}
With no value, the default is restored. With |widest*|, the string is 
built using |<integer>| as the value of the counter (e.g., with 
\verb|\roman|, 
\verb|widest=viii| and \verb|widest*=8| are the same).

Since |\value| does not return a string but a number, 
|widest| and the \verb|*| values cannot be used with it. 
However, with \verb|widest*|, being a number, it is allowed.

\subsection{More on horizontal spacing}

Since |\parindent| is not used as such inside lists, but instead
is set internally to either |\itemindent| or |\listparindent|,
when used as the value of a parameter \textsf{enumitem} returns the
global value, i.  e., the value it has outside the outermost list.

The horizontal space in the left margin of the current level is
distributed in the following way:\footnote{Admittedly, these figures
are not exactly the clearest possible, and I intend to improve them in
a future release}
\begin{center}
\begin{tabular}{cc}
\fbox{\fbox{\strut \texttt{labelindent}}
  \fbox{\strut \texttt{labelwidth}}
  \fbox{\strut \texttt{labelsep} $-$ \texttt{itemintent}}}
&
\fbox{\strut\texttt{itemindent}}\\
\texttt{leftmargin}
\end{tabular}
\end{center}

\begin{desc}
\verb|labelsep*=<length>|\3
\end{desc}

Remember |labelsep| spans part of |leftmargin| and
|itemindent| if the latter is not zero.  This is often somewhat
confusing, so a new key is provided---with \texttt{labelsep*} the
value is reckoned from the left margin (it just sets |\labelsep| 
and then adds |\itemindent| to it, but in addition later changes to
|itemindent| are taken into account):
\begin{center}
\begin{tabular}{cc}
\fbox{\fbox{\strut \texttt{labelindent}}
  \fbox{\strut \texttt{labelwidth}}
  \fbox{\strut \texttt{labelsep*}}}
&
\fbox{\strut\texttt{itemindent}}\\
\texttt{leftmargin}
\end{tabular}
\end{center}

\begin{desc}
|labelindent*=<length>|\3
\end{desc}

Like |labelindent|, but it is reckoned from the left margin in 
the current list and not from that in the enclosing list/text.

\subsection{Numbering, stopping, and resuming}

\begin{desc}
|start=<integer>|
\end{desc}
Sets the number of the first item.

\begin{desc}
|resume|
\end{desc}

The counter continues from the previous |enumerate|,
instead of being reset to 1.
\begin{verbatim}
\begin{enumerate}
\item First item.
\item Second item.
\end{enumerate}
Text.
\begin{enumerate}[resume]
\item Third item 
\end{enumerate}
\end{verbatim}

This is done locally. If you want global resuming, see next section on series.

\begin{desc}
|resume*|
\end{desc}

Like |resume| but the options from the previous list are used,
too.  This option must be restricted to the optional argument in a
environment (this is the only place where it makes sense).  It should
be used sparingly---if you are using it often, then very likely you
want to define a new list (see \ref{s.clone}).  Further keys are allowed, and
in this case the saved options are overriden by those in the current
list (i.e., the position of \texttt{resume*} does not matters).  For
example:
\begin{verbatim}
\begin{enumerate}[resume*,start=1] % or [start=1,resume*]
\end{verbatim}
uses the keys in the previuos \texttt{enumerate}, but restarts the
counter.  If there is a series of a certain list with
\texttt{resume*}, options are taken from the list previous to the
first one, except for \texttt{start}.

\subsection{Series}

\begin{desc}
|series=<series-name>|\3\\
|<series-name>|\qquad|resume*=<series-name>|
\qquad|resume=<series-name>|\3
\end{desc}

A new method (3.0) of continuing lists is by means of the key
\texttt{series}, so that they behave like a unit.  A list with key
\texttt{series} is considered the starting list and its settings are
stored \textit{globally}, so that they can be used later with
\texttt{resume}/\texttt{resume*}.  All these keys take a value with
the series name (which must be different from existing keys):
\begin{itemize}
\item |resume=<series-name>| just continue numbering items in the
series,
\item |resume*=<series>| also applies the settings of the 
starting list,
\item |<series>|, i.e., the series name used as a key, is an 
alternative to |resume*=<series>|.
\end{itemize}
For example:
\begin{verbatim}
\begin{enumerate}[label=\arabic*(a),leftmargin=1cm,series=lafter]
\item A
\item B
\end{enumerate}
\end{verbatim}
You get: 1(a)  2(a). You can continue with:
\begin{verbatim}
\begin{enumerate}[label=\arabic*(b),resume*=lafter]
                 % or [label=\arabic*(b),lafter]
\item A
\item B
\end{enumerate}
\end{verbatim}
You get: 3(b)  4(b). (But you can use |start=1|, if you like.)

Note you can add further arguments, which are executed after those
saved at the starting list and therefore take precedence over them --
in particular, |resume*| itself takes precedence over a
|start| (e.g., |start=1|) in the the starting list.

Every time a series is started, several commands are defined 
internally, so to avoid wasting resources and use the same name for 
non-overlapping series.


\subsection{Penalties}

\begin{desc}
|beginpenalty=<integer>|\qquad
|midpenalty=<integer>|\qquad |endpenalty=<integer>|
\end{desc}

Set the penalty at the beginning of a list, between items and at the
end of the list, respectively.  Please, refer to your \LaTeX{} or
\TeX{} manual about how penalties control page breaks.  Unlike other
parameters, when a list starts their values are not reset to the
default, thus they apply to the child lists.
\begin{desc}
|before=<code>| \qquad |before*=<code>|
\end{desc}

Execute code before the list starts (more precisely, in the second
argument of the |list| environment used to define them).  The
unstarred form sets the code to be executed, overriding any previous
value, while the starred one adds the code to the existing one (in
the setting hierarchy, see below, \textit{not} with relation to the
enclosing list/text).  It can contain, say, rules and text, but this
has not been extensively tested.  All calculations have been finished,
and you can access and manipulate the list parameters.  For example,
to have both margins (left and right) set to the widest label:
\begin{verbatim}
\setlist{leftmargin=*,before=\setlength{\rightmargin}{\leftmargin}}
\end{verbatim}

\begin{desc}
|after=<code>|\qquad|after*=<code>|
\end{desc}

Same, but just before the list ends.

\subsection{Description styles}

A key available in |description|.
\begin{desc}
|style=<name>|
\end{desc}

Sets the description \textit{style}. |<name>| can be any of the 
following:
\begin{itemize}

\item |standard|: like |description| in standard classes, although
with other classes it could be somewhat different.  The label is
boxed.  Sets \verb|itemindent=!|.

\item |unboxed|: much like the standard |description|, but 
the label is not boxed to avoid uneven spacing and unbroken labels if 
they are long. Sets \verb|itemindent=!|.

\item |nextline|: if the label does not fit in the margin, the text
continues in the next line, otherwise it is placed in a box of width
|\leftmargin| $-$ |\labelsep|, i.e., the item body never sticks into
the left margin.  Sets \verb|labelwidth=!|.

\item |sameline|: like |nextline| but if the label does not 
fit in the margin the text continues in the same line. Same as
\verb|style=unboxed,labelwidth=!|.

\item |multiline|: the label is placed in a parbox whose width is 
|leftmargin|, with several lines if 
necessary. Same as \verb|style=standard,align=parleft,labelwidth=!|.

Three caveats: (1) mixing boxed and unboxed labels has not a
well-defined behaviour, (2) when nesting list all combinations are
allowed but not all make sense, and (3) nesting \verb|nextline| lists
is not supported (it works, but its behaviour might change in the
future, because the current one is not what one could expect).



\end{itemize}


\subsection{Compact lists}

\begin{desc}
|noitemsep|\qquad|nosep|
\end{desc}

The key |noitemsep| kills the space between items and paragraphs
(i.e., |itemsep=0pt| and |parsep=0pt|), while
|nosep| kills all vertical spacing.\footnote{The key 
\texttt{nolistsep}, now deprecated, introduced a thin stretch, which 
was not the intended behaviour.}

\subsection{``Wide'' lists}

\begin{desc}
\verb|wide|\3\\
\verb|wide=<parindent>|
\end{desc}

With this convenience key, the leftmargin is null and the label is
part of the text---in other word, the items look like ordinary
paragraphs.\footnote{\texttt{fullwidth} is deprecated.} Here |labelsep|
sets the separation between the label and the first word.  It is
equivalent to
\begin{verbatim}
align=left, leftmargin=0pt, labelindent=\parindent,
listparindent=\parindent, labelwidth=0pt, itemindent=!
\end{verbatim}
With |wide=<parindent>| you may set at once another value instead of
|\parindent|.  Of course, these keys can be overriden after
\verb|wide|, too; for example, remembering that with left-aligned labels
the text is pushed if the they are wider than |labelwidth|, you
can set |labelwidth=1.5em| for a minimal width, or instead of
|itemindent=!| you may prefer |itemindent=*|, which sets the
minimal width to that of widest label.  In level 2 you may prefer
|labelindent=2\parindent|, and so on.  You may also want to
combine it with |noitemsep| or |nolistsep|.

\section{Inline lists}
\label{s.inline}

\3

Inline lists are ``horizontal'' lists set as ordinary text inside a
paragraph.  With this package you can create inline lists, as
explained below, with \verb|\newlist|, which have their own labels and
counters.  However, in most cases inline versions of standard lists,
with the same labelling schema, will be enough -- the package option
\verb|inline| does that.


\begin{desc}
|inline| \qquad(package option)\\
\texttt{enumerate*}\qquad\texttt{itemize*}\qquad
\texttt{description*} \qquad(environments)
\end{desc}

With the package option \texttt{inline}, three environments for inline
lists are defined: \texttt{enumerate*}, \texttt{itemize*}, and
\texttt{description*}.  They emulate the behaviour of
\textsf{paralist} and \textsf{shortlst} in that labels and settings
are shared with the displayed (ie, ``normal'') lists \texttt{enumerate},
\texttt{itemize} and \texttt{description}, respectively (however,
remember resuming is based on environment names, not on list types).  This applies
only to those created with \texttt{inline} -- inline lists created
with |\newlist| as described below are independent and use their
own labels and settings. Note as well \verb|inline| is not required 
if you needn't inline versions of standard lists.

\begin{desc}
|itemjoin=<string>|\qquad|itemjoin*=<string>|
\qquad|afterlabel=<string>|
\end{desc}

Format is set with keys \texttt{itemjoin} (default is a space), and
\texttt{afterlabel} (default is |\nobreakspace|, ie, |~|).
An additional key is \texttt{itemjoin*}, which, if set, is used
instead of \texttt{itemjoin} before the last item.  So, with
\begin{verbatim}
before=\unskip{: }, itemjoin={{; }}, itemjoin*={{, and }}
\end{verbatim}
the following punctuation between items is used:
\begin{quote}
Blah blah: (a) one; (b) two; (c) three, and (d) four. Blah blah
\end{quote}

\verb|itemjoin| is ignored in vertical mode (i.e., in mode unboxed
and just after a quote, a displayed list and the like).

\begin{desc}
|mode=unboxed|\qquad|mode=boxed|
\end{desc}

Items are boxed, so floats are lost and nested lists are not allowed
(remember many displayed elements are defined as lists).  If using
floats or lists inside inline lists is important, use an alternative
``mode'', which you can activate with
\texttt{mode=unboxed} (the default is \texttt{mode=boxes}).  With it
floats may be used freely, but misplaced |\item|s are not catched and
\texttt{itemjoin*} is ignored (a warning is written to the log about
this fact).

\section{Global settings}

Global changes, to be applied to all of these list, are also
possible:
\begin{desc}
|\setlist[enumerate,<levels>]{<format>}|\\
|\setlist[itemize,<levels>]{<format>}|\\
|\setlist[description,<levels>]{<format>}|\\
|\setlist[<levels>]{<format>}|
\end{desc}
Where |<level>| is the list level (one or more) in |list|, and the
corresponding levels in |enumerate| and 
|itemize|.\footnote{\verb|\string\setenumerate|, 
\verb|\string\setitemize| and \verb|\string\setdescription| are 
deprecated.}  With no
|<levels>|, the format applies to all of them.  Here list does not
mean any list but only the three ones handled by this package and
those redefined by this package or defined with |\newlist| (see
below).  For example:
\begin{verbatim}
\setlist{noitemsep}
\setlist[1]{\labelindent=\parindent} % << Usually a good idea
\setlist[itemize]{leftmargin=*}
\setlist[itemize,1]{label=$\triangleleft$}
\setlist[enumerate]{labelsep=*, leftmargin=1.5pc}
\setlist[enumerate,1]{label=\arabic*., ref=\arabic*}
\setlist[enumerate,2]{label=\emph{\alph*}),
                      ref=\theenumi.\emph{\alph*}}
\setlist[enumerate,3]{label=\roman*), ref=\theenumii.\roman*}
\setlist[description]{font=\sffamily\bfseries}
\end{verbatim}
These setting are read in the following order: list, list at the
current level, enumerate/itemize/description, and
enumerate/itemize/description at the current level; if a key appears
several times with different values, the last one, i.e.,  the most
specific one, is applied.  If we are resuming a series or a list with
\texttt{resume*}, the saved keys are then applied.  Finally, the
optional argument (except \texttt{resume*}), if any, is applied.

\LaTeX{} provides a set of macros to change many of these parameters,
but setting them with the package is more consistent and sometimes
more flexible at the cost of being more ``explicit'' (and verbose).

The list specification can contain variables and counters, provided
they are expandable, and counters are \textsf{calc}-savvy, so that if
you load this package you can write things like:
\begin{verbatim}
\newcount{toplist}
\setcount{toplist}{1}
\newcommand{\mylistname}{enumerate}
\setlist[\mylistname,\value{toplist}+1]{labelsep=\itemindent+2em]
\end{verbatim}
This allows defining lists within loops.

Currently, a way to discriminate the font size is not provided
(|\normalsize|, |\small|\dots).

\section{\textsf{enumerate}-like labels}
\label{s.short}

\begin{desc}
|shortlabels| (package option)
\end{desc}

With the package option \texttt{shortlabels} you can use an
\textsf{enumerate}-like syntax, where |A|, |a|, |I|,
|i| and |1| stand for |\Alph*|, |\alph*|,
|\Roman*|, |\roman*| and |\arabic*|.  This is intended
mainly as a sort of compatibility mode with the \textsf{enumerate}
package, and therefore the following special rule applies: if the very
first option (at any level) is not recognized as a valid key, then it
will be considered a label with the \textsf{enumerate}-like syntax.  For
example:
\begin{verbatim}
\begin{enumerate}[i), labelindent=\parindent]
...
\end{enumerate}
\end{verbatim}
Although perhaps not so useful, you can omit |label=| in the
itemize environment under similar conditions, too:
\begin{verbatim}
\begin{itemize}[\textbullet]
...
\end{itemize}
\end{verbatim}

\begin{desc}
|\SetEnumerateShortLabel{<key>}{<replacement>}|
\end{desc}

With this command, you can define new keys (or redefine them), which is
particularly useful for enumerate to be adapted to especific
typographical rules or to extend it for non-Latin scrips. Here
|<replacement>| contains one of the starred versions of 
counters. For example:
\begin{verbatim}
\SetEnumerateShortLabel{i}{\textsc{\roman*}}
\end{verbatim}
redefines |i| so that items using this key are numbered with
small caps roman numerals. The key has to be a single letter.

\section{Cloning the basic lists}
\label{s.clone}

\begin{desc}
|\newlist{<name>}{<type>}{<max-depth>}|\\
|\renewlist{<name>}{<type>}{<max-depth>}|
\end{desc}

The three lists can be cloned so that you can define ``logical''
environments behaving like them.  To define a new lists (or redefine a
existing one), use |\newlist| (or |\renewlist|), where |<type>| is
|enumerate|, |itemize| or |description|.

If <type> is |enumerate|, a set of counters with names |<name>i|,
|<name>ii|, |<name>iii|, |<name>iv|, etc.  (depending on <max-depth>)
is defined.  Don't use an arbitrarily large number for <max-depth>, to
avoid creating too many counters.  Then you can use those counters in
labels; e.  g., if you have defined a list named \texttt{steps}, you
can define a label with:
\begin{verbatim}
label=\arabic{stepsii}.\arabic{stepsi}
\end{verbatim}

\begin{desc}
\verb|\setlist[<names>,<levels>]{<keys/values>}|\\
\verb|\setlist*[<names>,<levels>]{<keys/values>}|
\end{desc}

After creating a list, you can (in fact you
must, at least the label) set the new list with |\setlist|:
\begin{verbatim}
\newlist{ingredients}{itemize}{1}
\setlist[ingredients]{label=\textbullet}
\newlist{steps}{enumerate}{2}
\setlist[steps,1,2]{label=(\arabic*)}
\end{verbatim}
Names in the optional argument of |\setlist| say which lists applies the
settings to, and numbers say the level (it is calc-savvy).  Several
lists and/or several levels can be given, and all combinations are
set; e.g.:
\begin{verbatim}
\setlist[enumerate,itemize,2,3]{...}
\end{verbatim}
\noindent sets enumerate/2, enumerate/3, itemize/2 and itemize/3. 
No number (or 0) means ``all levels'' and no name means ``all lists''; no 
optional argument means ``all lists at all levels''.

The three inline lists have types \texttt{enumerate*},
\texttt{itemize*}, and \texttt{description*}, which are available
always, even without the package option |inline| (which just defines
three environments with these names).

The starred form \verb|\setlist*| adds the settings to previous ones.

\begin{desc}
\verb|\setlistdepth{<integer>|\3
\end{desc}

By default, \LaTeX{} has a limit of 5 nesting levels, but when 
cloning list this value may be too short, and therefore you may want 
to set a new value. In levels below the 5th (or the deepest defined by a 
class), the settings of the last are used (i.e., \verb|\@listvi|).

\section{More about counters}

\subsection{New counter representation}

\begin{desc}
\verb|\AddEnumerateCounter{<LaTeX command>}{<internal command>}{<widest label>}|
\end{desc}

``Registers'' a counter representation so that \textsf{enumitem}
recognizes it.  Intended mainly for non Latin scripts, but also useful
in Latin scripts.  For example:
\begin{verbatim}
\makeatletter
\def\ctext#1{\expandafter\@ctext\csname c@#1\endcsname}
\def\@ctext#1{\ifcase#1\or First\or Second\or Third\or
Fourth\or Fifth\or Sixth\fi}
\makeatother
\AddEnumerateCounter{\ctext}{\@ctext}{Second}
\end{verbatim}
A starred variant allows to give a number instead of a string as the 
widest label; for example, if the widest label is that corresponding 
to the value 2:\3
\begin{verbatim}
\AddEnumerateCounter*{\ctext}{\@ctmoreext}{2}
\end{verbatim}
This variant is to be preferred if the representation is not a plain 
string but it is styled, e.g.,
with small caps. (The counter names can contain |@| even if not a letter.) 

\subsection{Restarting \texttt{enumerate}s}

\begin{desc}
|\restartlist{<list-name>}|\3
\end{desc}

Currently, with
\begin{verbatim}
\setlist[enumerate]{resume}
\end{verbatim}
you can get a continuous numbering through a document.  A new command has
been added for restarting the counter in the middle of the document:
\begin{verbatim}
\restartlist{<list-name>}
\end{verbatim}

It is based solely in the list name, not the list type, which means
\texttt{enumerate*} as defined with the package option \texttt{inline}
is not the same as \texttt{enumerate}, because its name is different.

\section{Generic keys and values}

\begin{desc}
|\SetEnumitemKey{<key>}{<replacement>}|\3
\end{desc}

With this command you can create your own (valueless) keys.  For
example:
\begin{verbatim}
\SetEnumitemKey{midsep}{topsep=3pt,partopsep=0pt}
\end{verbatim}

Keys so defined can then be used like the others. Another example is
multicolumn lists, with \textsf{multicol}:
\begin{verbatim}
\SetEnumitemKey{twocol}{
  itemsep=1\itemsep,
  parsep=1\parsep,
  before=\raggedcolumns\begin{multicols}{2},
  after=\end{multicols}}
\end{verbatim}

(The settings for \texttt{itemsep} and \texttt{parsep} kill the
stretch and shrink parts.  Of course, you may want to define a
new list.)

Note the package may introduce new keys in the future, so
\verb|\SetEnumitemKey| is a potential source of forward
incompatibilities.  However, it's safe using a non-letter character
other than hyphen or star in the key name (e.g., \verb|:name| or 
\verb|2_col|).

\begin{desc}
|\SetEnumitemValue{<key>}{<string-value>}{<replacement>}|\3
\end{desc}

This commands provides a further abstraction layer for the
|<key>=<value>| pairs.  With it you can define logical names which
are translated to the actual value.  For example, with:
\begin{verbatim}
\SetEnumitemValue{label}{numeric}{\arabic*.}
\SetEnumitemValue{leftmargin}{standard}{\parindent}
\end{verbatim}
you might say:
\begin{verbatim}
\begin{enumerate}[label=numeric,leftmargin=standard]
\end{verbatim}
So, you can left to the final design what |label=numeric| means.

\section{Package options}

Besides \verb|inline|, \verb|ignoredisplayed|, and \verb|shortlabels|,
the following option is available.

\begin{desc}
\verb|loadonly|
\end{desc}

With this package option the package is loaded but the three
lists are not redefined. You can create your own lists, yet, or
even redefine the existing ones.

\section{Three patterns}

Three list layouts could be considered very
frequent. Let us apply the parameters above to define them. (Below 
are samples.)

The first pattern aligns the label with the surrounding
|\parindent| while the item body is indented depending
on the label and a fixed |labelsep|:
\begin{verbatim}
labelindent=\parindent,
leftmargin=*
\end{verbatim}
A fairly frequent variant is aligning the label with the
surrounding text (rememeber |labelindent| is |0pt| by
default):
\begin{verbatim}
leftmargin=*
\end{verbatim}
The former looks better in the first level while the latter
seems preferable in subsequent ones. That can be easily
set with
\begin{verbatim}
\setlist{leftmargin=*}
\setlist[1]{labelindent=\parindent} % Only the level 1
\end{verbatim}

The second pattern aligns the item body with the surrounding
|\parindent|. In this case:
\begin{verbatim}
leftmargin=\parindent
\end{verbatim}

A third pattern would be to align the label with |\parindent|
and the item body with |2\parindent|:
\begin{verbatim}
labelindent=\parindent,
leftmargin=2\parindent,
itemsep=*
\end{verbatim}
Again, a variant would be to align the label with the surrounding
text and the itembody with |\parindent|:
\begin{verbatim}
leftmargin=\parindent,
itemsep=*
\end{verbatim}

Note here |\parindent| means the global value applied
to normal paragraphs.

\section{The trivlist issue}

\LaTeX{} uses a simplified version of |list| named |trivlist| to set
displayed material, like |center|, |verbatim|, \verb|tabbing|,
\verb|theorem|, etc., even if conceptually they are not lists.
Unfortunately, |trivlist| uses the current list settings, which has
the odd side effect that changing the vertical spacing of lists also
changes sometimes the spacing in these environments.

This package modifies |trivlist| so that the default settings for 
the current level (ie, those set by the corresponding |clo| 
files) are set again. In standard \LaTeX{} that is usually redundand, 
but if we want to fine tune lists, not resetting the default values 
could be a real issue (particularly if you use the |nolistsep| 
option).

A minimal control of vertical spacing has been made possible 
with\footnote{\verb|\string\setdisplayed| is deprecated.}
\begin{itemize}
\item |\setlist[trivlist,<level>]{<keys/values>}|
\end{itemize}
but |trivlist| itself, which is not used directly very
often, does not accept an optional argument. This feature
is not intended as a full-fledge |trivlist| formatter.

If for some reason you do not want to change |trivlist|
and preserve the original definition, you can use the
package option |ignoredisplayed|.

\section{Samples}

\expandafter\ifx\csname setenumerate\endcsname\relax

Please, install first the package and then typeset this
document again.

\else

In these samples we set |\setlist{noitemsep}|

\setlist{noitemsep}
\small

\newcommand{\newsample}{\vskip6pt\goodbreak\hrule height 1pt\vskip6pt}
\newcommand{\samplesep}{\vskip6pt\goodbreak\hrule\vskip6pt}
\newbox\vsep
\setbox\vsep\hbox{\vrule height 2ex depth 16ex width 1pt}
\dp\vsep0pt
\newcommand\showsep{\leavevmode\llap{\copy\vsep}}

\newsample

\begin{verbatim}
En un lugar de la Mancha, de cuyo nombre no quiero acordarme,
no ha mucho tiempo que viv\'{\i}a un hidalgo de los de
\begin{enumerate}[labelindent=\parindent,leftmargin=*]
  \item lanza en astillero,
  \item adarna antigua,
  \item roc\'{\i}n flaco, y
  \item galgo corredor.
\end{enumerate}
Una olla de algo m\'{a}s vaca que carnero, salpic\'{o}n las m\'{a}s
noches, duelos y quebrantos los s\'{a}bados...
\end{verbatim}

The rule shows \verb|labelindent|. 

\samplesep

\showsep En un lugar de la Mancha, de cuyo nombre no quiero acordarme,
no ha mucho tiempo que viv\'{\i}a un hidalgo de los de
\begin{enumerate}[labelindent=\parindent,leftmargin=*]
\item lanza en astillero,
\item adarna antigua,
\item roc\'{\i}n flaco, y
\item galgo corredor.
\end{enumerate}
Una olla de algo m\'{a}s vaca que carnero, salpic\'{o}n las m\'{a}s
noches, duelos y quebrantos los s\'{a}bados...

\newsample

With |\begin{enumerate}[leftmargin=*] % labelindent=0pt by default|. 

The rule shows \verb|labelindent|.

\samplesep

\noindent\showsep\hskip\parindent En un lugar de la Mancha, de cuyo nombre no quiero acordarme,
no ha mucho tiempo que viv\'{\i}a un hidalgo de los de
\begin{enumerate}[leftmargin=*]
\item lanza en astillero,
\item adarna antigua,
\item roc\'{\i}n flaco, y
\item galgo corredor.
\end{enumerate}
Una olla de algo m\'{a}s vaca que carnero, salpic\'{o}n las m\'{a}s
noches, duelos y quebrantos los s\'{a}bados...

\newsample

With |\begin{enumerate}[leftmargin=\parindent]|.

The rule shows \verb|leftmargin|.

\samplesep

\showsep En un lugar de la Mancha, de cuyo nombre no quiero acordarme,
no ha mucho tiempo que viv\'{\i}a un hidalgo de los de
\begin{enumerate}[leftmargin=\parindent]
\item lanza en astillero,
\item adarna antigua,
\item roc\'{\i}n flaco, y
\item galgo corredor.
\end{enumerate}
Una olla de algo m\'{a}s vaca que carnero, salpic\'{o}n las m\'{a}s
noches, duelos y quebrantos los s\'{a}bados...

\newsample

With |\begin{enumerate}[labelindent=\parindent,|\allowbreak
| leftmargin=*,|\allowbreak| label=\Roman*.,|\allowbreak
| widest=IV,|\allowbreak| align=left]|.

The rule shows \verb|labelindent|.

\samplesep

\showsep En un lugar de la Mancha, de cuyo nombre no quiero acordarme,
no ha mucho tiempo que viv\'{\i}a un hidalgo de los de
\begin{enumerate}[labelindent=\parindent, leftmargin=*,
                  label=\Roman*., widest=IV, align=left]
\item lanza en astillero,
\item adarna antigua,
\item roc\'{\i}n flaco, y
\item galgo corredor.
\end{enumerate}
Una olla de algo m\'{a}s vaca que carnero, salpic\'{o}n las m\'{a}s
noches, duelos y quebrantos los s\'{a}bados...

\newsample

With |\begin{enumerate}[label=\fbox{\arabic*}]|. A reference to
the first item is \ref{i:first}

\samplesep

En un lugar de la Mancha, de cuyo nombre no quiero acordarme,
no ha mucho tiempo que viv\'{\i}a un hidalgo de los de
\begin{enumerate}[label=\fbox{\arabic*}]
\item \label{i:first}lanza en astillero,
\item adarna antigua,
\item roc\'{\i}n flaco, y
\item galgo corredor.
\end{enumerate}
Una olla de algo m\'{a}s vaca que carnero, salpic\'{o}n las m\'{a}s
noches, duelos y quebrantos los s\'{a}bados...

\newsample

With nested lists.

\samplesep

\begin{verbatim}
En un lugar de la Mancha, de cuyo nombre no quiero acordarme,
no ha mucho tiempo que viv\'{\i}a un hidalgo de los de
\begin{enumerate}[label=(\alph*), labelindent=\parindent,
     leftmargin=*, start=12]
\item lanza en astillero,
\begin{enumerate}[label=(\alph{enumi}.\roman*), leftmargin=*, start=7]
\item adarna antigua,
\end{enumerate}
\item roc\'{\i}n flaco, y
\begin{enumerate}[label=(\alph{enumi}.\roman*), leftmargin=*, resume]
\item galgo corredor.
\end{enumerate}
\end{enumerate}
Una olla de algo m\'{a}s vaca que carnero, salpic\'{o}n las m\'{a}s
noches, duelos y quebrantos los s\'{a}bados...
\end{verbatim}

En un lugar de la Mancha, de cuyo nombre no quiero acordarme,
no ha mucho tiempo que viv\'{\i}a un hidalgo de los de
\begin{enumerate}[label=(\alph*), labelindent=\parindent,
     leftmargin=*, start=12]
\item lanza en astillero,
\begin{enumerate}[label=(\alph{enumi}.\roman*), leftmargin=*, start=7]
\item adarna antigua,
\end{enumerate}
\item roc\'{\i}n flaco, y
\begin{enumerate}[label=(\alph{enumi}.\roman*), leftmargin=*, resume]
\item galgo corredor.
\end{enumerate}
\end{enumerate}
Una olla de algo m\'{a}s vaca que carnero, salpic\'{o}n las m\'{a}s
noches, duelos y quebrantos los s\'{a}bados...

\newsample

\begin{verbatim}
En un lugar de la Mancha, de cuyo nombre no quiero acordarme,
no ha mucho tiempo que viv\'{\i}a un hidalgo de los de
\begin{description}[font=\sffamily\bfseries, leftmargin=3cm,
    style=nextline]
  \item[Lo primero que ten\'{\i}a el Quijote] lanza en astillero,
  \item[Lo segundo] adarna antigua,
  \item[Lo tercero] roc\'{\i}n flaco, y
  \item[Y por \'{u}ltimo, lo cuarto] galgo corredor.
\end{description}
Una olla de algo m\'{a}s vaca que carnero, salpic\'{o}n las m\'{a}s
noches, duelos y quebrantos los s\'{a}bados...
\end{verbatim}

\samplesep

En un lugar de la Mancha, de cuyo nombre no quiero acordarme,
no ha mucho tiempo que viv\'{\i}a un hidalgo de los de
\begin{description}[font=\sffamily\bfseries, leftmargin=3cm,
    style=nextline]
\item[Lo primero que ten\'{\i}a el Quijote] lanza en astillero,
\item[Lo segundo] adarna antigua,
\item[Lo tercero] roc\'{\i}n flaco, y
\item[Y por \'{u}ltimo, lo cuarto] galgo corredor.
\end{description}
Una olla de algo m\'{a}s vaca que carnero, salpic\'{o}n las m\'{a}s
noches, duelos y quebrantos los s\'{a}bados...

\newsample

Same, but with |sameline|.

\samplesep

En un lugar de la Mancha, de cuyo nombre no quiero acordarme,
no ha mucho tiempo que viv\'{\i}a un hidalgo de los de
\begin{description}[font=\sffamily\bfseries, leftmargin=3cm,
    style=sameline]
\item[Lo primero que ten\'{\i}a el Quijote] lanza en astillero,
\item[Lo segundo] adarna antigua,
\item[Lo tercero] roc\'{\i}n flaco, y
\item[Y por \'{u}ltimo, lo cuarto] galgo corredor.
\end{description}
Una olla de algo m\'{a}s vaca que carnero, salpic\'{o}n las m\'{a}s
noches, duelos y quebrantos los s\'{a}bados...

\newsample

Same, but with |multiline|. Note the text overlaps if
the item body is too short.

\samplesep

En un lugar de la Mancha, de cuyo nombre no quiero acordarme,
no ha mucho tiempo que viv\'{\i}a un hidalgo de los de
\begin{description}[font=\sffamily\bfseries, leftmargin=3cm,
    style=multiline]
\item[Lo primero que ten\'{\i}a el Quijote] lanza en astillero,
\item[Lo segundo] adarna antigua,
\item[Lo tercero] roc\'{\i}n flaco, y
\item[Y por \'{u}ltimo, lo cuar] galgo corredor.
\end{description}
Una olla de algo m\'{a}s vaca que carnero, salpic\'{o}n las m\'{a}s
noches, duelos y quebrantos los s\'{a}bados...

\fi

\normalsize

\section{Afterword}

\subsection{\LaTeX{} lists}

As it is well known, \LaTeX{} predefines three lists:
\texttt{enumerate}, \texttt{itemize} and \texttt{description}.  This
is a very frequent classification which can also be found in, say,
HTML. However, there is a more general model based in three
fields---namely, label, title, and body---, so that enumerate and
itemize has label (numbered and unnumbered) but no title, while
description has title but no label.  In this model, one can have a
description with entries marked with labels, as for example (of 
course, this simple solution is far from satistactory):
\begin{verbatim}
\newcommand\litem[1]{\item{\bfseries #1,\enspace}}
\begin{itemize}[label=\textbullet]
\litem{Lo primero que ten\'{\i}a el Quijote} lanza en astillero,
... etc.
\end{verbatim}

\vskip6pt
\goodbreak
\hrule
\vskip6pt

\newcommand\litem[1]{\item{\bfseries #1,\enspace}}
En un lugar de la Mancha, de cuyo nombre no quiero acordarme,
no ha mucho tiempo que viv\'{\i}a un hidalgo de los de
\begin{itemize}[label=\textbullet]
\litem{Lo primero que ten\'{\i}a el Quijote} lanza en astillero,
\litem{Lo segundo} adarna antigua,
\litem{Lo tercero} roc\'{\i}n flaco, y
\litem{Y por \'{u}ltimo, lo cuarto} galgo corredor.
\end{itemize}

\vskip6pt
\goodbreak
\hrule
\vskip6pt

This format in not infrequent at all and a tool for defining them is 
on the way and at a very advanced stage. It has not been include in 
version 3.0 because I'm not sure if the proper place is this package 
or \textsf{titlesec} and it is not stable enough yet.

\subsection{Known issues}

\begin{itemize}

\item
List resuming is based on environment names, and when a
|\newenvironment| contains a list you may want to use |\begin| and
|\end|.  Using the corresponding commands, however, is not an error,
but it is your responsability to make sure the result is correct.

\item
It seems there is no way to catch a misspelled name in |\setlist|
and a meaningless error ``Missing number, treated as zero'' is raised.

\item The behaviour of mixed boxed labels (including enumerate and
itemize) and unboxed labels is not well-defined.  The same applies to
boxed and unboxed inline lists (which could even raise an error).
Similarly, resuming a series and a list at the same time is allowed,
too, but again its behaviour is not well-defined.

\item (3.5.2) An incompatibility with 2.x has popped up -- if you were
using the optional argument to pass a value to a \verb|\ref| or other 
macro requiring expandable macros, an error is raised. A quick fix 
is letting \verb|\makelabel| to |\descriptionlabel| in \texttt{before}.

\end{itemize}

\subsection{What's new in 3.0}

\begin{itemize}
\item Inline lists, with keys to set how items are joined (ie, the
punctuation between items).  Two modes are provided: \verb|boxed|  and
\verb|unboxed|.

\item \verb|\setlist| is \textsf{calc}-savvy (eg, for use in loops),
and you can set diferent lists and levels at once.  \item All lengths
related to labels can take the value \verb|*| (and not only
\verb|labelsep| and \verb|leftmargin|).  Its behaviour has been made
consistent and there is new value \verb|!| which does not compute the
widest label.

\item With \verb|\restartlist{<list-name>}|, list counters can be restarted (in
case you are using \verb|resume|).

\item \verb|resume*| can be combined with other keys.

\item Lists can be gathered globally using series, so that they are
considered a single list.  To start a series just use
|series=<series-name>| and then resume it with |resume=<series-name>| or
|resume*=<series-name>|.

\item The ``experimental'' \verb|fullwidth| has been replaced by a new key
\verb|wide|.

\item|\SetLabelAlign| defines new align values.

\item You can define ``abstract'' values (eg, \verb|label=numeric|) and
new keys.
\end{itemize}

\begin{itemize}
\item (3.2) \verb|start| and \verb|widest*| are \textsf{calc}-savvy.
\item (3.2) \verb|\value| can be used with \verb|widest*|.
\item (3.2) Some internal restrictions in \verb|\arabic| and the like
has been removed.  It is more flexible at the cost of having a more
``relaxed'' error checking.
\end{itemize}
\subsection{Bug fixes}

\begin{itemize}
\item Star values (eg, \verb|leftmargin=*|) could not be overriden
and new values were ignored.
\item \verb|nolistsep| as the first of several keys was not always
recognized and therefore treated like a short label
(i.e., \verb|nol\roman*stsep|).
\item \verb|labelwidth| did not always work (when there was a prior
\verb|widest| and \verb|*|)
\item With \verb|align=right| the label and the following text could
overlap.
\item \verb|description| did not get the correct list level.
\item At some point (2.x?) \verb|\value*| stopped working.
\item (3.1) Unfortunately, \textsf{xkeyval} ``kills'' 
\textsf{keyval}, so the lattest has been replicated in 
\textsf{enumitem}.
\item (3.3) Fixes a serious bug -- with \verb|*| neither 
\verb|itemize| nor \verb|description| worked.
\item (3.4) Fixes bad spacing in mode boxed (misplaced \verb|\unskip| 
before the first item and wrong spacefactor between items).
\item (3.4) \verb|nolistsep| did not work as intended, but since the
error has been there for several years, a new key \verb|nosep| is
provided.
\item (3.4) The issue with \verb|nolistsep| with \verb|shortlabels| 
(see above) was not fixed in all cases. Hopefully now it is.
\item (3.5.0) Fixed the fix related to the spacefactor between items.
\item (3.5.0) Fixed a problem with nested boxed inline lists.
\item (3.5.1) \texttt{resume*} only worked once, and subsequent ones
bahaved like \texttt{resume}.
\item (3.5.2) Fixed |\setlist*|, which didn't work.
\end{itemize}


\subsection{Acknowledgements}

I wish to thank particularly the comments and suggestions from Lars
Madsen, who has found some bugs, too.

\end{document}
